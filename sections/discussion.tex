\chapter{Discussion}
\label{ch:discussion}

In this chapter, the results that were presented in the previous chapter are analyzed. The reasons for the performance differences but also the similarities of the agents are explained in Section \ref{sec:discussion_analysis}. Furthermore, Section \ref{sec:limitations} elaborates on limitations of the chosen solution approach.

\section{Analysis of the results}
\label{sec:discussion_analysis}

\subsection{Hyperparameter optimization}

The results for the hyperparameter optimization are particularly interesting for two reasons: First, the extended experiments are based on the resulting hyperparameters that are deemed best. Second, the results are quite different for the three agents. 

A combination of a relatively low exploration constant and a shallow search horizon leads to the best results for the agent with the full unweighted action set. This combination has the effect that the agent constructs a search tree with a maximum depth of only five actions, while it explores relatively little. The agent with the reduced action space yields the best result with a combination of a shallow search horizon but a large exploration constant. During planning, it also only considers five potential future actions but is inclined to explore considerably more. The agent using the full action space starts to exploit early during planning on actions that have shown to be rewarding initially. In contrast, the agent with a reduced action space considers actions more often before starting to exploit on the ones which show better returns. The difference in the agents' behaviors makes sense considering the different number of actions they can choose from during planning. The agent utilizing the full action space has to be more selective. Otherwise, it would \emph{waste} a large amount of its searches to explore the many potential actions and would not be able to generate a meaningful belief at any node.

% TODO: Provide concrete planning time

For the agent using preferred actions, The best results are achieved with a deep search horizon and a low exploration constant. The agent plans ahead 25 actions into the future, with relatively little exploration. The increased search horizon has the effect that the planning complexity, and thereby the planning time, is significantly greater than for the other two agents. It is very important to take this into account. The potential applicability to driving with a real human is highly dependent on the planning time.

\subsection{Convergence behavior}

All agents can lead to good policies if they perform a high number of searches. For the simple driver model, convergence occurs significantly earlier for all agents than with the less predictable driver models. For the simple driver model, all three agents perform quite similar. The agent using the full unweighted action space converges first but the other two agents reach slightly higher results later on. From 1500 searches onward, for the agents using the full action space, both weighted and unweighted, the terminal states reached during runs are caused by particle divergence. The agent using the reduced action set encounters terminal states because it cannot counteract a distracted driver's wrong actions, precisely because its action space is reduced.

For the more complex driver model implementing steering overcorrection, using preferred actions leads to a substantially earlier convergence and to fewer terminal states. The performance of the agent using preferred actions converges somewhere around 1500 searches. From this point on, no terminal states are reached anymore. The other two agents still lead to terminal states during some runs for all evaluated number of searches. The return for episodes where no terminal state is reached is similar to the simple driver model experiments. The agents driving policies are therefore not worse. What is worse is their ability to estimate the current state of the driver. The terminal states result almost exclusively from particle deprivation after a formerly distracted driver becomes attentive again and overcorrects. This was expected since the complexity of the planning task increases with the introduction of oversteering. Only the agent using preferred actions is able to avoid particle deprivation.

The experiments for the third driver model with steering overcorrection and additional driver action noise, lead to similar resuls as the experiments without noise. The most likely reason for this is the discretization applied to the driver's actions (see Section~\ref{sec:discretization}). Low noise that is added to an action is often lost during discretization. The resulting action can therefore be the same action as it would have been before adding the noise. However, there are cases where a combination of a strong overcorrection with a high driver action noise leads to unexpected situations and therefore particle deprivation, even for the agent with preferred actions.

% TODO: Instead of random action selection, the agent shoudl stop acting when particle deprivation occurs. The random action selection was chosen because this is the way the experiments in the original POMCP paper are designed, but it would make more sense to disable the assistance altogether.

% TODO: Discuss performance impact of high number of searches explicitly
% The planning time would be very useful for this!

\paragraph{Main problem: Particle deprivation}

The main cause for suboptimal behavior and reaching terminal states is particle deprivation. The more complex the problem, the more searches are needed to guarantee that a wide array of possible future scenarios are covered. As long as their belief represents the environment's and driver's states well, all agents lead to a good lane keeping performance. However, even with a large number of searches, the agents without preferred actions reach terminal states in some runs, independent of which driver model is used. The only agent that consistently avoids reaching terminal states in all experiments with more than 5000 searches is the agent using preferred actions. 

Particle deprivation occurs when an agent's belief strongly diverges from the true state. At some point, none of the observations it receives during planning will match the real observation anymore. The agent has lost track of the true state completely. For the node representing the real observation, the belief tree stores no particles. From this point on, the agent is effectively clueless about the true state and continues to use random action selection.

Estimating the car's state is not problematic As the car states in the initial belief are very similar to the true car state, and the simulation engine is deterministic (same state and action always lead to the same next state), the car states in the belief do not diverge significantly from the true state during the experiments. The problem lies in the estimation of the state of the driver model. If the agent wrongfully beliefs that the driver is attentive or distracted and the real observations match the observations from planning for some time, the agent's belief will converge to the wrong state. Particle injection (see Section \ref{sec:particle_deprivation} is implemented to circumvent this to some degree. However, the method is no guaranteed cure for particle deprivation.

Using preferred actions lowers the chance of particle deprivation. Due to the use of preferred actions, the agent does not start with equal initial values at new nodes during rollouts. Instead, the actions are weighted with domain knowledge (see Section~\ref{sec:preferred_actions}). The likelihood of selecting an action for a rollout is bound to its intensity, with less severe steering actions being preferred as the need for strong steering is scarce on a highway track. Assuming the underlying assumption of the introduced domain knowledge is valid, and the return is confirmed to be higher for a preferred action during initial searches, then exploration is kept to a minimum. If the reward does not drop sufficiently, the agent is allowed to exploit on the preferred action. It is like lowering the threshold of trustworthyness for preferred actions; they need less initial confirmation than others. Thereby, a preferred  action, if successful initially, is evaluated relatively often, even with fewer searches. Consequently, nodes connected with the preferred actions hold a more comprehensive belief. More driver model states will be considered and therefore particle deprivation is prevented.

Intuitively, one would expect the agent with the reduced action set to lead to a similar improvement when it comes to particle deprivation as the agent using preferred actions. The reduced number of actions means that the fixed number of searches is distributed over fewer actions. Thereby, more potential driver models could be covered per action, which should reduce the chance of particle deprivation. However, during the experiments the agent with the reduced action set did not lead to significantly lower terminal states reached due to particle deprivation than the agent using the full action set. A plausible explanation for this could be the large exploration constant for the agent with the reduced action set. The high value for the exploration constant leads to intensive exploration. Thereby it could counteract the effect of the reduced action space. An additional test with a smaller exploration constant would be necessary to verify this assumption. However, this test has not been performed as part of this thesis.

% TODO: Mention deep search horizon and potential effects
% Does looking far into the future help further?

% Particle deprivation - Are there possible countermeasures?

\subsection{Mean lane centeredness}

% TODO: Better explanation for variance of preferred actions agent
% TODO: Comparison to baseline would be good to back up claims
Increasing the number of searches during plnning leads to better lane centering for all agents. This was to be expected as the ability to judge the driver's state correctly increases with the number of searches. Moreover, the higher the number of searches, the lower seems t be the variation. With 10000 searches, the car can mosly be kept between 0.5 and 1.0 meters from the lane center, which is a promising result. The agent with preferred actions performs better than the others, but exhibits a higher variance in its lane centeredness at 10000 searches. A possible explanation could be that it accounts better for the times it has to purposefully deviate from the center of the lane, in order to achieve better values subsequently. However, further experiments would be necessary to verify this assumption.

\section{Limitations}
\label{sec:limitations}

There are multiple factors that limit the applicability of the solution approach taken in this thesis to a more realistic setting or even a real world scenario. The most striking limitations of the solution approach are elaborated in the following subsections.

\subsection{Long planning time}

% TODO: Here, I REALLY need concrete times.

One of the most striking shortcoming of the online solution approach taken in this thesis is the long planning time the agents require for the forward search they perform during planning. Anything beyond a few fractions of a second would be too much in a real world driving scenario. The agents that are analyzed in this thesis require many searches during planning in order to avoid particle deprivation. Only the agent using preferred action was able to consistently avoid terminal states for all driver models, and that only with 7500 searches or more. The other agents may require even more forward searches. Especially for the agent with preferred actions, due to the deep search horizon, performing a large number of searches means having to plan for a long time. Performing 7500 searches with a search horizon of 25 means the agent has to simulate a total of 187500 actions.

It may be possible to speed up the computation and therefore reduce the amount of planning time that is needed. There are two main ways in which improvements can be made. First, the generative model can be made more efficient. Second, the POMCP algorithm could be parallelized or potentially even replaced by a more efficient algorithm. Both options are discussed in Section~\ref{sec:perf_opt}. 

\subsection{Dependency on a reliable generative model}

For the experiments conducted in this thesis, the same driving simulator and driver model that were used as a simulation of a real driving scenario, where also used as the generative model for th agent. The transition and observation probabilities underlying the generative model were therefore an exact replication of the dynamics of the agent's environment (car and driver). POMCP is based on the assumption that such a generative model exists which can sample the true transition and observation probabilities. However, in a real-life scenario, or even just while using a more realistic driver model, such a generative model might not be available. It might be possible to use a generative model that just approximates the dynamics of the environment. This has not been evaluated and likely leads to problems if the approximation is not very accurate.

\subsection{Action and observation space discretization}
% Cannot handle continuous driver actions

The action and observation space for the agents are discrete. POMCP is not suited to be used with continuous action and observation spaces (see Section \ref{sec:discretization}). Steering actions and a car's sensory information are naturally continuous. The results indicate that a discretization of the observation space can be successful. However, the more limited the number of bins used for the discretization, the lower is the precision. On the one hand, increasing the resolution of the discretization can lead to a considerable elevation of the planning complexity. More distinct observations may be encountered while constructing the search tree. A low precision of the observations, on the other hand, does not allow the agent to act precisely, leading to suboptimal behavior. Limited steering precision is also caused by discretizing the action space. If the agent can only choose from a limited number of actions, it cannot steer accurately. Its assistance might be jumpy. This does not allow for a smooth driving experience. Section \ref{sec:conclusion-continuous} discusses briefly how this obstacle could be overcome. 

% TODO: Add a potential explanation
Furthermore, the actions of the driver are discretized. In any realistic scenario, this just could not be the case as it reduces the driver's precision which is obviously unacceptable. The state space is continuous. Therefore, theoretically, POMCP can work with continuous driver actions, as long as they are discretized for the observations. However, none of the agents is remotely successful if continuous driver actions are used. It becomes impossible to account for all potential driver actions while planning. A wide array of sampled actions would probably be enough to allow for a good estimation. However, a low discretization resolution for the observations leads to a big range of different driver actions that are represented by one observation. Planning becomes too imprecise. Using a continuous observation space could therefore potentially enable the agent to plan with continuous driver actions as well. Further experiments would be necessary to verify this assumption.

\subsection{Driver does not learn or adapt}

Research suggests that drivers adapt their driving behavior when they are assisted by a lane keeping assistant \parencite{behavior_adapt}. The driver model used in the experiments does not evolve over time. It does not adapt to the behavior of the agent. Even if the agent would steer into the wrong direction multiple times in a row, the driver model does not adapt to the agent by adjusting its behavior and steering stronger to acocunt for the agent's wrong actions. The driver also does not develop any reliance on the agent - it does not reduce its steering efforts, trusting the agent to correct it, like it would be conceivable with real humans.

% TODO: Add Observations are not noisy, which is unrealistic

% TODO: Mention? Speaks for bad experiment design
% \subsection{Edge cases}

% Human-Robot Mutual Adaptation in Shared Autonomy