\chapter{Abstract}
\label{sec:abstrac}

Driver assistance systems are paving the way for automated driving. Until fully autonomous driving will be available and in wide-spread use, assistance systems, such as lane-keeping assistance, can help prevent accidents by supporting a human driver. However, if a lane-keeping assistance system strongly restricts the driver's autonomy, the driver may overly trust the system \parencite{over-trust} and distract herself more frequently \parencite{driver-distraction}. It is advantageous to base the intensity of assistance on the attentiveness of the driver (see \cite{disracted-lane-keeping-1} and \cite{disracted-lane-keeping-2}). Thereby, the driver is kept in the loop when attentive but is supported during periods of distraction. Driver distraction is a serious issue; 14\% of crashes in the USA were affected by distracted driving in 2018 \parencite{distracted_nhtsa}. Taking into account the driver's distraction for the activation of assistance technology could help prevent such accidents.

We design an agent as a lane-keeping assistant that shares control of the vehicle with the driver. The driver's distraction is estimated online, allowing the agent to assist a distracted driver while keeping an attentive driver in control. For the estimation of the driver's distraction, the agent relies solely on driving performance measures, such as the driver's steering movements and sensory information about the vehicle's position. To account for uncertainty about the driver's distraction and the exact position of the vehicle, the problem is modeled as a \acrfull{pomdp}. To the best of our knowledge, this is the first study using only commonly available driving performance metrics instead of sophisticated driver monitoring systems to estimate the driver's distraction with a POMDP.

We apply the \acrfull{pomcp} algorithm \parencite{pomcp} to solve the POMDP online. The algorithm performs Monte-Carlo tree search, sampling possible future scenarios to form a strategy. Our experiments confirm that the driver's overall lane-keeping performance is significantly enhanced. Our approach has potential. However, there are obstacles that have to be overcome for the method to be viable in practice. First, the solver is not efficient enough; planning takes too much time. Second, we use simple hand-crafted driver models for our experiments. The driver model can and should be replaced by a more sophisticated and realistic model. Third, our method relies on the discretization of the action and observation spaces. A car's sensory information and steering actions are naturally continuous. We discuss these limitations in detail and provide suggestions for future improvements.


% \begin{enumerate}
%     \item How can a lane-keeping scenario with shared control between a potentially distracted human driver and an agent be modeled using a POMDP?
%     \item Can the agent estimate the driver's distraction using driver performance measures alone, allowing it to take appropriate actions when the driver becomes distracted?
%     \item Is the solution approach viable for a real-world scenario, and if not, what limitations are there, and what are potential methods to solve them?
% \end{enumerate}

% \begin{enumerate}
%     \item We provide a POMDP model with a continuous state as a representation of the shared control lane-keeping scenario, outlining how both the human driver and the car's dynamic can be simulated. 
%     \item Our model for the human driver is simple. However, our modeling approach is also suitable for the integration of a more sophisticated driver model (see section \ref{sec:complex-driver}).
%     \item We enable an agent to act as a lane-keeping assistant to the driver, taking into account the driver's potential distraction. The POMDP is solved online by applying the POMCP algorithm. Experimental results show that the driving performance is enhanced.
%     \item Particle deprivation is a common problem with a particle filter approach such as POMCP. Implementing particle injection (see section \ref{sec:particle_deprivation}) and introducing domain knowledge by the use of preferred actions (see section \ref{sec:preferred_actions}) leads to an improvement.
%     \item Using the TORCS driving simulator as a generative model during planning with POMCP is not efficient enough for a real-time scenario. The performance needs to be significantly optimized. Suggestions on how to achieve this are provided in section \ref{sec:perf_opt}.
%     \item The lane-keeping performance of our approach is inferior to traditional lane-keeping assistance systems. The immediate application of the approach is not advisable. Section \ref{sec:future} outlines opportunities for improvement.
%     \item Our simulation method allows for a repetition of experiments. Problems can be revisited and analyzed. This is important in the safety-critical domain of automated driving.
% \end{enumerate}