\chapter{Experimental setup}
\label{sec:setup}

% TODO: Define terminal state
% TODO: Define initial state
% TODO: Mention seed based on run

% TODO: Refine
The task for the agent in the experiment is to keep the car centered in a highway lane. Thus, the track used for the agent's evaluation needs to represent such a scenario. Most tracks readily available in the racing car simulator TORCS are race tracks. These are much wider than common roads and the width often differs in different segments. To ensure a realistic scenario, a one-lane track with a continuous width of 3.5m, which is common for European roads, is used. The track covers a wide array of scenarios. It includes long straight segments, both left and right curves, and multiple curves of alternating directions in a row. By ensuring that all common highway scenarios are covered by the track, a single track is sufficient.

The car used for the experiment does not have a big impact on driving performance, as long as it is consistent during all experiments. To ensure that an action's effect at a particular position are consistent, the speed of the car is constant.

The driver is pulled for an action every 0.1 seconds. The simulation tick rate is 0.002 seconds. When the driver is not pulled, its last action is repeated. It follows, that every action is repeated during 50 simulation ticks. The simulation is not in real time. Therefore, the simulation waits for the agent's planning. If the agent is pulled for an action, the environment does not change until the agent's next action is decided and performed.

\section{Evaluated scenarios}
\subsection{Driver model}
\subsection{Action space and action selection}
\section{Design decisions}
% TODO Leave out? Or add something useful here

\section{Hyperparameter optimization}

% TODO: Add parameter and hyperparameter table

% TODO: Refine
There are a number of hyperparameters. Most importantly, there is the planning time. This is the time the agent is allowed to search in and expand its search tree in order to find the most likely current state and best course of action. More planning time can result in a wider and/or deeper tree. The search horizon is limited by a discount threshold. If this threshold is reached, the search is stopped and no more actions will be performed for the current trajectory and, if there is enough planning time left, a new state is sampled from the current belief and a new planning trajectory is expanded. Moreover, there is an exploration constant. This value, determined before the start of the experiment, assigns actions that have not been tried before more expected reward and thus favors exploration.
\subsubsection{Number of searches}
\subsubsection{Exploration constant}
\subsubsection{Discount horizon}

\section{Performance metrics}
Due to the randomness involved in the driver model, each experiment run will lead to a different scenario. Therefore, to get credible results, the experiment has to be repeated many times. The average discounted return over all experiment runs serves as performance metric. This result is compared with the average reward of an agent that always performs the optimal reaction to the action of the driver model. The closer the POMCP agent's reward is to the optimal agent's reward, the better was the planning.

