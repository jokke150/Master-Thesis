\chapter{Introduction}
\label{sec:intro}

\section{Motivation}

Fully autonomously driving cars have the potential to rule out human driving error. However, many social and technical obstacles have yet to be overcome until fully autonomous cars become market-ready \parencite{autonomous_driving_book}. Cars will likely not be able to drive fully autonomously in the near future, continuing to require some degree of human supervision and intervention (\cite{human-needed}, and \cite{human-needed-2}). Until fully autonomous cars are available, driving assistance systems with increasing autonomy will be employed transitionally \parencite{autonomy-future}. Specific tasks, such as lane-keeping or cruise control, are delegated to an assistance system, while the driver retains the responsibility for other tasks.

% Before the introduction of fully autonomous driving, two transitional stages are conceivable \parencite{shared_control}: First, partially autonomous driving refers to the partial relinquishment of the responsibility of the driver. Only specific tasks, such as lane-keeping or cruise control, are delegated to an assistance system, while the driver retains the responsibility for other tasks. Second, conditional automation goes one step further; the car is controlled autonomously. Only if intervention by the driver is deemed necessary, the driver is alerted, and control is handed over. 

However, delegating tasks to an assistance system could limit the driver's autonomy. A loss of autonomy can turn driving into a monotonous and tedious supervisory task. As a result, drivers may distract themselves more frequently \parencite{driver-distraction}, have increased reaction times \parencite{reaction-time}, suffer from a lower situational awareness \parencite{situational-awareness}, and can place too much trust in the assistance system \parencite{over-trust}. Moreover, it may take the driver too long to take over control from the assistance system if necessary in critical situations \parencite{takeover-time}. To mitigate these issues, \cite{shared-control-haptics} propose employing a shared control scheme: The driver and the assistance system continuously share the control authority of the car, cooperating in the driving task. The driver is kept in the loop. Thereby, the driving task stays enjoyable, and drivers are prevented from relying too heavily on the assistance system.

A large part of driving accidents is associated with driver distraction \parencite{distracted_nhtsa}. Therefore, it seems reasonable to make the extend of the assistance dependent on the driver's level of attention. Whenever a driver is inattentive or distracted, an assistance system needs to be particularly sensitive. \cite{disracted-lane-keeping-1} and \cite{disracted-lane-keeping-2} show that increasing the intensity of lane-keeping assistance when the driver is distracted increases lane-keeping performance, while the driver is not influenced in her driving while being attentive. 

We propose a novel method for lane-keeping assistance with a potentially distracted human in the loop. The presented assistance system serves as a proof-of-concept to account for the uncertainty about the driver's distraction and exact car position in a shared-control lane-keeping scenario. The system plans ahead, estimating the driver's attentiveness using a driver model. In contrast to previous approaches, we do not rely on intricate measures such as gaze monitoring or head tracking to classify the driver's attentiveness. Planning is based on the driver's past steering actions alone. We use a \Gls{pomdp} to model the decision process and account for the uncertainty involved. The POMDP is solved online using the \Gls{pomcp} algorithm \parencite{pomcp}.

\section{Related work and contributions}

Lane-keeping assistance systems have been a topic of strong scientific research interest and development \parencite{lane-keeping-road-boundary-detection}. Most car manufacturers today already offer lane-keeping assistance systems \parencite{lane-keeping-widely-used}. An in-depth survey of approaches is out of scope for this thesis (see \cite{driver-assistance-review-2}, and \cite{driver-assistance-review} for an overview). Instead, we focus on lane-keeping assistance systems that take the driver's distraction into account. To the best of our knowledge, none of the commercially available lane-keeping assistants adapt to the driver's attentiveness. 

\cite{disracted-lane-keeping-1} and \cite{disracted-lane-keeping-2} study the effect of adapting the intensity of lane-keeping assistance based on the driver's distraction. The concept is promising. The method, if the detection of the driver's distraction is accurate, proves to be effective at keeping the driver engaged when she is attentive while keeping a distracted driver safe at the same time.

Many methods have been proposed to classify driver distraction using methods that use video recordings to classify the driver's attentiveness. Methods like eye movement tracking \parencite{eye-movement}, head tracking \parencite{head-tracking}, and driver gaze monitoring \parencite{gaze-monitoring} have been proven to be successful. However, they all rely on continuous video recordings and analysis of the driver. This constitutes a privacy intrusion and may not be desirable or impractical (for example, in low-light situations). 

Research suggests that driver distraction can lead to a change in driving performance. Distracted drivers show certain behaviors, such as lane position deviations, infrequent steering wheel movements, and increased reaction times \parencite{driver-distraction-review}. \cite{dist-det-perf} show that monitoring driving performance measures (steering wheel angle, heading angle, and lateral deviation) alone can be sufficient to detect driver distraction accurately. Driven by those findings, we propose to estimate the driver's attentiveness in an online manner based on on driver's past driving behavior alone.

We use a \acrfull{pomdp} to account for the uncertainty about the driver's distraction. POMDP is a framework to model sequential decision problems (one decision influences the next) in a partially observable environment \parencite{pomdp-definition}. An agent interacts with the environment. After performing an action, the agent receives observations about the environment and a reward signal. It uses the observations it receives over time to estimate the environment's state. Using its estimate of the true state of the environment, the agent can decide which actions to perform next to maximize the long-term reward.

POMDPs are well suited to model driving problems in which the internal psychological state of a human is important but unknown (such as her intend, goals, drowsiness, or attentiveness). Research mostly focuses on fully autonomous cars assessing the behavior of other traffic participants. By employing a POMDP to model the uncertainty about humans' internal states, improved driving policies were achieved in three, widely studied scenarios: pedestrian avoidance (\cite{despot-crowd}, and \cite{pomdp-pedestrian-avoid-2}), intersection crossing (\cite{pomdp-intersection}, \cite{att_intersec}, \cite{pomdp-intersection-2}, and \cite{pomdp-intersection-3}), and lane changes (\cite{pomdp-lane-changes}, \cite{att_intersec}, \cite{pomdp-lane-changes-2}, \cite{tactical-decision}, and \cite{pomdp_towards_human}). 

The aforementioned studies demonstrate that a POMDP is suitable for dealing with the uncertainty related to a person's internal psychological state and that this can be beneficial for making the right decisions when driving. However, the focus lies on estimating the state of external drivers rather than considering the state of the driver in a shared control scenario. \cite{hitl_pomdp} introduce a framework to model shared control driving scenarios with a human in the loop using POMDPs. The benefits of the framework are exemplified by assisting a potentially drowsy driver with lane-keeping: The assistance system is able to reason about the driver's drowsiness effectively and can handle noisy or erroneous observations. A driver model is used to simulate human behavior, and a simple car model is employed to simulate the car's dynamics in the experiments. However, the approach relies on the availability of observations that strongly correlate with the driver's drowsiness (observation whether the driver's eyes are closed or not).

In this thesis, an assisted lane-keeping scenario with a potentially distracted human in the loop is modeled as a POMDP and solved online using the POMCP algorithm. Our method bases the detection of driver distraction on the driver's past actions alone. No sophisticated or closely correlated measures are required. The driving dynamics are simulated using a realistic driving simulator. Labeled data is not required. Our method is unsupervised and based on a driver model. As far as we know, this is the first work to use online POMDP solving to account for uncertainty about a driver's distraction in a shared-control, lane-keeping scenario while relying only on commonly available driving performance metrics as observations.

\section{Problem overview and research questions}

The problem examined in this thesis is assisted lane keeping with shared control by a potentially distracted human driver and an agent acting as an assistance system. The agent assists the driver in keeping the car centered in its lane. The driver becomes distracted periodically. During distraction episodes, the driver may act suboptimally. The driver's behavior is simulated using a driver model. The driver’s attentiveness is unknown to the agent. It has to infer if the driver is distracted from the driver's past steering activity and sensory observations about the car's state. The dynamics of the car are simulated using the driving simulator \Gls{torcs}. A POMDP model is employed to account for the uncertainty about the driver's attentiveness.

\vspace{1em}
\noindent
\textbf{The following research questions arise:}
\begin{enumerate}
    \item How can a lane-keeping scenario with shared control between a potentially distracted human driver and an agent be modeled using a POMDP?
    % \item Which simplifications are necessary in order to address the problem?
    % \item How can a good policy for the agent be derived from the POMDP?
    % \item How does the agent perform with driver models of increasing complexity?
    % \item What measures are available to improve the agents performance?
    \item Can the agent estimate the driver's distraction using driver performance measures alone, allowing it to take appropriate actions when the driver becomes distracted?
    \item Is the solution approach viable for a real-world scenario, and if not, what limitations are there, and what are potential methods to solve them?
\end{enumerate}

\section{Outline}

The remaining content of the thesis is organized as follows:
\begin{description}
    \item[\cref{ch:theory}] introduces the basic theoretical \gls{pomdp} concepts which serve as a foundation for the thesis.

    \item[\cref{ch:problem}]
    presents the methodology, including the definition of the POMDP used to model the shared control lane-keeping task and the chosen solution approach.
    
    \item[\cref{ch:setup}]
    describes the experiments that we perform to assess the performance of the solution approach.
    
    \item[\cref{ch:results}]
    showcases the results from the experiments.

    \item[\cref{ch:discussion}]
    analyzes the results and elaborates the limitations of the solution approach.

    \item[\cref{ch:discussion}]
    highlights the conclusion, contributions, and challenges for a potential future real-world application.

\end{description}