\chapter{Introduction}
\label{sec:intro}

\iffalse

Driver   inattention: inattention  occurs  when  the  driver’s  allocation  of resources  to  activities  does not  match  the  demands  of  activities  required  for  the control of safety margins.

Driver distraction: where  the  driver  allocates  resources  to a  non-safety critical activity while the resources allocated to activities critical for safe driving do not match the demands of these activities.

Types of distraction:
    Visual: taking your eyes off the road
    Manual: taking your hands off the wheel
    Cognitive: taking your mind off driving2                             

DISTRACTION

Visual distraction or cognitive distraction have been investigated by combining vehicle state [36], [147], [148], drivers’ visual state [145]–[147], [149], and operations [147], [149], [191]. Answers to the question of how to measure a driver’s cognitive distraction have been given in [150]. Learning-based approaches such as deep sparse autoencoders [192], deep belief networks or DBNs [188], support vector machines (SVM) [145], [193] have been widely used to detect and classify driver distraction. \cite{shared_control}


COGNITIVE MODEL

One of the most utilized means is based on the “adaptive control of thought-rational (ACT-R) [141]”
cognitive where the discrete nature of drivers’ control actions is captured from a cognitive perspective. For example, Salvucci et al. developed an integrated cognitive pathfollowing driver model [142] and lane-change driver model [143] using the combination of the ACT-R cognitive architecture and perceptual-motor process. \cite{shared_control}



It is well known that manual control is prone to human errors. On the other hand, fully automated tasks are currently subject to wide-ranging limitations in decision-making and situationawareness. To exploit full potentials of both of human and automation while overcoming the barriers of car-to-driver transition, Mulder et al. [27] presented an entirely different control scheme – shared control systems3. The human driver and the automated driving agent continuously share and cooperatively complete a specific driving task, thereby allowing drivers to enjoy driving while keeping in control consistently. Moreover, the shared-control scheme can synergize innate human capacities and technological capabilities to enable us to realize our full potential [31].



Keeping the lane is one of the primary tasks of controlling a vehicle. Especially on longtrips  on  extra-urban  roads  this  might  be  a  monotonous  and  annoying  taskwhere unintended lane departures may occur caused  by momentary lapses  ofattention or drowsiness. As a consequence of an unintended lane departure theremight happen a
●Collision with a stationary object
●Collision with a vehicle traveling in the same direction
●Collision with oncoming traffic●Rollover accident
●Collision with pedestrians and bicyclists beside the road
●Further accident caused by failed corrective steering and braking (loss of vehiclecontrol)

Generally speaking unintended lane departures are one major cause of severeaccidents. However, there might be various root causes of an unintended lane depar-ture like driver distraction, drowsiness, a temporary blackout of the driver, toohigh velocity (especially in curves), poor visibility of lane markers and road geometry,and more. These chains of cause and effect need to be considered for defining thescope  of  lane  keeping  and  lane  departure  warning  systems  as  well  as  theireffectiveness

https://link.springer.com/referenceworkentry/10.1007%2F978-0-85729-085-4_26

Theconcept of Honda LKA systems is based on a cooperative operation between driver andvehicle intended to lighten the operation load but at the same time not to diminish drivermotivation (Ishida et al.2003).


the road to hell is paved with good intentions…
https://delfthapticslab.nl/project/responsible-adaptation-of-lane-keeping-assistance/

How can we assist drivers in lane-keeping, without inducing them to misuse this assistance to drive faster? This question relates to behavioral adaptation: a central concept in the design of assistance systems that we know in daily life as “the road to hell is paved with good intentions…”. It states that when we design systems that support humans, they often adapt their behavior in a way that mitigates the very benefits that the system aimed to realize. Advanced driver assistance systems (ADAS) are a well-known example of designs were anticipated safety benefits are diminished because drivers show undesired behavioral adaptations. Literature shows that ADAS lead to increased risk-taking, such as driving at a higher speed, driving closer to a lead vehicle, or performing distractive non-driving tasks[1].

\fi

%
% Overview first: Tell a story
%

Fully autonomously driving cars have the potential to rule out human driving error which is at least a contributing factor to most accidents today. Many social and technical obstacles have yet to be overcome until fully autonomous cars become market-ready \parencite{autonomous_driving_book}. However, many \glspl{adas} such as adaptive cruise control, lane keeping and changing assistance, and automated collision mitigation are already deployed in modern cars. 

The extend to which an \gls{adas} takes control varies. While the potential prevention of human-error caused accidents increases with the elaborateness of intervention by an assistant system, excessive intervention drastically limits the driver's autonomy. A loss of driver autonomy can turn driving into a monotonous and tedious supervisory task. Drivers easily become inattentive and are more prone to distract themselves, for example by looking on their phone. However, as long as assistance systems are not sufficient to handle all situations, a concentrated human will remain necessary to take actions in situations the assistance system fails. Leaving the driver with a pure supervision task can lead to a long transition time for the driver when it is required to retake control of the vehicle \parencite{shared_control}. Being in control means having to concentrate. Therefore, the goal should be to keep the human driver in control as much as possible but to assist when help is really needed. As a result, driving pleasure is enhanced and drivers are prevented from relying too heavily on the assistance systems.

% Human drivers and \gls{adas} often already \emph{share} the control over the car.

% The extend to which an \gls{adas} takes control varies. Some systems only suggest actions to the driver, others actively intervene in steering, acceleration, or braking, and semi-automated systems assume full authority over the car's control, leaving the driver only with the task of supervision. While the potential prevention of human-error caused accidents increases with the elaborateness of intervention by an \gls{adas}, excessive intervention drastically limits the driver's autonomy. 

% Leaving the driver with a pure supervision task can lead to over-trust, neglect and complacency and may result in a long transition time for the driver when it is required to retake control of the vehicle \parencite{shared_control}. 

% Shared control of the car by a human driver and an agent acting as \gls{adas} allows to exploit both the human's and the agent's unique qualities. 

% TODO: Add source for claim of different behavior when inattentive?
15\% of injury crashes in the US were associated with driver distraction in 2018 \parencite{distracted_nhtsa}. Therefore, it seems reasonable to make the extend of the \gls{adas}'s activation dependent on the driver's level of attention. Whenever a driver is inattentive or distracted, an \gls{adas} needs to be particularly sensitive. Yet in what way can an assistance system detect that a driver is distracted? 


% TODO: Add literature to back up eye tracking machine learing claim
There have bee attempts to develop systems that determining the psychological state of a driver in real time while driving. The application of eye tracking technology or analysis of camera footage using machine learning models is conceivable and has led to promising results. However, as promising as promising as these methods are, they are not readily available yet. Furthermore, they are quite intrusive and could be seen as an encroachment on privacy. Thus, the driver's level of attention is essentially unknown. Nevertheless, one can assume that distracted drivers act differently. Among other things, deviations such as increased reaction times and altered steering behavior are likely.

% If any of clues are noticeable, the system needs to lower its activation threshold.

% Besides  the  per-ception,  autonomous  driving  systems  constitute  of  multipletasks where classical supervised learning methods are no moreapplicable.  First,  when  the  prediction  of  the  agent’s  actionchanges  future  sensor  observations  received  from  the  envi-ronment under which the autonomous driving agent operates,for  example  the  task  of  optimal  driving  speed  in  an  urbanarea.  Second,  supervisory  signals  such  as  time  to  collision(TTC),  lateral  error  w.r.t  to  optimal  trajectory  of  the  agent,represent  the  dynamics  of  the  agent,  as  well  uncertainty  inthe  environment.  Such  problems  would  require  defining  thestochastic cost  function to be  maximized. Third, the  agent isrequired  to  learn  new  configurations  of  the  environment,  aswell  as  to  predict  an  optimal  decision  at  each  instant  whiledriving in its environment. This represents a high dimensionalspace given the number of unique configurations under whichthe agent & environment are observed, this is combinatoriallylarge. In all such scenarios we are aiming to solve a sequential decision  process,  which  is  formalized  under  the  classicalsettings  of  Reinforcement  Learning  (RL),  where  the  agent  isrequired  to  learn  and  represent  its  environment  as  well  asact  optimally  given  at  each  instan


A two-fold problem arises: On the one hand, a lane keeping assistance system has to be able to identify when drivers are distracted by observing their behavior. On the other hand, the system must have the capability to provide meaningful assistance. 



Intuitively, it seems reasonable to solve both problems individually; using a model that takes the available data, such as the driver's steering behavior, as input to classify whether the driver is distracted, and another model that assists a distracted driver in steering the car. Both could be trained using example data. However, this supervised approach entails two challenges: First, driving is a sequential decision process. An action influences future actions and driving situations in which decisions have to be made are essentially unique. Second, an activation of the assistance system can affect on how drivers behave. Drivers may adjust to the system. It is not possible to create a dataset that covers these dynamics entirely.


%to learn
%the differences in driving styles between attentive and distracted drivers from example data using supervised machine learning and then use this insight to conditionally intensify an \gls{adas} assistance. 



\Gls{rl} allows a system to learn and represent its behavior by interacting with it rather than learning from past experience. Therefore, \gls{rl} constitutes a promising method to develop an \gls{adas} or even a fully autonomous driving agent and its application in this area is a very active research area with many successful results \parencite{rl_driving_survey}. Because learning is achieved by exploration rather than from examples, \gls{rl} is able to perform well in sequential decision making tasks. Moreover, reinforcement learning algorithms can be extended to support learning with a partially observable state \parencite[p.~466]{RL_introductio}. While the agent can perceive the car's environment with sensors, the attention level of the driver is hidden. Nevertheless, only one \gls{rl} agent is needed to both learn how to assist in driving and to classify when this is desired due to a distracted driver.

The result is a shared control scenario where both the human driver and the agent can actively control (e.g. steer, brake, accelerate) the car simultaneously. Each can indirectly perceive the actions of the other by observing the state of the car. Thereby, on the one hand, the agent is able to analyze the driving behavior of the human and, on the other hand, the human can notice the assistance of the agent and may adapt to it. 

% TODO: interaction loop

% Offline: specify, prior to the execution, the best action to execute for all possible situations.

% While these approximate algorithms can achieve very good performance, they often take significant time (e.g. more than an hour) to solve large problems, where there are too many possible situations to enumerate (let alone plan for). Furthermore, small changes in the environment’s dynamics require recomputing the full policy, which may take hours or days.

% Whereas an offline search would compute an exponentially large contingency plan considering all possible happenings, an online search only considers the current situation and a small horizon of contingency plans.



Learning in a real-world situation is not feasible in the context of this thesis. Despite the inevitable high safety risk, it would also require an enormous investment of resources, and the complexity of a real-world driving scenario represents an insurmountable obstacle. Instead, the agent learns in a simulation environment with a simulated human driver. \gls{torcs}, a racing car simulator that allows to model various driving situations \parencite{torcs} is used as simulation environment. It offers a good balance between realism and resource efficiency and has been utilized in many papers regarding \gls{rl}-based driving before. An \gls{act-r} cognitive model is employed to simulate the human's actions. The model is able to keep the car in its lane, perform lane changes, and avoid collision with other road users. It captures behavioral differences between attentive and inattentive human drivers. Furthermore, a human-subject experiment is performed in the \gls{torcs} simulation environment to identify if the agent is able to generalize well enough to be useful for actual human drivers.

The main goal and differentiator of this thesis is to utilize reinforcement learning for a shared-control driving task with unknown attention of the human driver. One of the main challenges is that near real-time decisions of the agent are necessary. This drastically limits the time available for online planning. Accordingly, the implementation needs to be very efficient. Solving the problem using an algorithm that requires discretized states (e.g. steering angle categories) is contrasted with a solution using an algorithm directly supporting continuous states.

\section*{Outline}

The rest of the proposal is organised as follows:
\begin{description}
    \item[\cref{sec:problem}]
    describes the problem in a formal manner using a \gls{pomdp}.
    
    \item[\cref{sec:literature}]
    summarizes and reviews important literature that serves as the foundation of the thesis.
    
    \item[\cref{sec:plan}]
    presents the initial research plan for the rest of the thesis, including important milestones and deadlines.
\end{description}


\iffalse

\cite{hitl_pomdp} define a similar \gls{pomdp} problem. An agent is supposed to activate a warning signal if the driver is drowsy and actively interfere by steering the car if the warning is unsuccessful in alerting the driver. The model representing the transition probabilities of the driver's interal state and action choice in \cite{hitl_pomdp} are arbitrarily constructed and likewise very simplified. A more realistic driving setting also requires a more sophisticated driver model. 


% An offline randomized point-based value iteration approach is employed to solve the POMDP for an approximate optimal policy. The policy is computed by iteratively sampling a finite set of random points from the agent's belief space. The agent thus interacts randomly with the environment in order to find an approximation of an optimal policy. 

% However, the state space is discretized and of low complexity. Solving large \gls{pomdp} requires the use of an online solver that plans from the current belief \parencite{online_pomdp_cont}. Not every point in the general belief space may be (easily) reachable, making some potential beliefs less important to consider for planning.

% Moreover, the model representing the transition probabilities of the driver's interal state and action choice in \cite{hitl_pomdp} are arbitrarily constructed and likewise very simplified. A more realistic driving setting also requires a more sophisticated driver model. 

\fi


% TODO: Decide:
% - discrete vs continuous state and action space
% - online or offline belief state planning
% - model based or model-free 
% - how simple must the scenario be



% Instead of limiting the state and action space to discrete values, continous states and actions are considered in this thesis. Moreover, 




% The driving scenario, and with it also the \gls{pomdp}, become significantly more complex. An online \gls{rl} algorithm is applied in order to learn the agent's policy.





% TODO: Goal and main distinction from prior work




% Advantages + Disadvantages: \cite{online_pomdp}

% Utilizing online \gls{pomdp} planning in a driving task with a hidden inner psychological state of a human driver constitutes a of this thesis. Most previous work focused on solving for an approximately optimal policy offline. 










