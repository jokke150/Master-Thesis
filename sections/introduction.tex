\chapter{Introduction}
\label{sec:intro}

\section{Motivation}

Fully autonomously driving cars have the potential to rule out human driving error, which is at least a contributing factor to most accidents today. However, many social and technical obstacles have yet to be overcome until fully autonomous cars become market-ready \parencite{autonomous_driving_book}. Cars will likely not be able to drive fully autonomously in the near future, continuing to require some degree of human supervision and intervention (\cite{human-needed}, and \cite{human-needed-2}). Until fully autonomous cars are available, increasinlgy autonomous driving assistance systems will be employed transitionally \parencite{autonomy-future}. Only specific tasks, such as lane-keeping or cruise control, are delegated to an assistance system, while the driver retains the responsibility for other tasks.

% Before the introduction of fully autonomous driving, two transitional stages are conceivable \parencite{shared_control}: First, partially autonomous driving refers to the partial relinquishment of the responsibility of the driver. Only specific tasks, such as lane-keeping or cruise control, are delegated to an assistance system, while the driver retains the responsibility for other tasks. Second, conditional automation goes one step further; the car is controlled autonomously. Only if intervention by the driver is deemed necessary, the driver is alerted, and control is handed over. 

However, delegating tasks to an assistance system could limit the driver's autonomy. A loss of autonomy can turn driving into a monotonous and tedious supervisory task. As a result, drivers may distract themselves more frequently \parencite{driver-distraction}, have reduced reaction times \parencite{reaction-time}, suffer from a lower situational awareness \parencite{situational-awareness}, and can place too much trust in the assistance system \parencite{over-trust}. Moreover, in the case of conditional automation, the time required for the driver to shift back to driving from a non-driving task may be too long in critical situations \parencite{takeover-time}. To mitigate these issues, \cite{shared-control-haptics} propose employing a shared control scheme: The driver and the assistance system continuously share the control authority of the car, cooperating in the driving task. The driver is kept in the loop. Thereby, the driving task stays enjoyable, and drivers are prevented from relying too heavily on the assistance systems.

A large part of driving accidents is associated with driver distraction \parencite{distracted_nhtsa}. Therefore, it seems reasonable to make the extend of the assistance dependent on the driver's level of attention. Whenever a driver is inattentive or distracted, an assistance system needs to be particularly sensitive. \cite{disracted-lane-keeping-1} and \cite{disracted-lane-keeping-2} show that increasing the intensity of lane-keeping assistance when the driver is distracted increases lane-keeping performance, while the driver is not influenced in her driving while being attentive. 

However, the detection of driver distraction is complex. Methods like eye movement tracking \parencite{eye-movement}, head tracking \parencite{head-tracking}, and driver gaze monitoring \parencite{gaze-monitoring} have been proposed and proven to be successful. However, they all rely on continuous video recordings and analysis of the driver. This constitutes a privacy intrusion and may not be desirable or impractical (for example, in low-light situations). 


%  In the introduction, we need to clarify 'what is the novelty of your research comparing with existing works in the car-driving assist agents'.


% Before those sentences, please add several sentences:
% - to summarize your main idea/proposal from a higher level. What is your key idea.
% Hopefully it can show the direct difference from the previous works 
% - to explain why it can get rid of the mentioned issue. 

This thesis presents an alternative approach: A \Gls{pomdp} is used to model a shared-control lane-keeping scenario. A driver is assisted by a lane-keeping assistance system that estimates the driver's attentiveness online, based on the driver's past behavior. Explicitly measuring the driver's distraction is not required.

% The novelty of our approach lies in the omission of the requirement of explicitly measuring the driver's distraction.

% To the best of our knowledge, this is the first work that applies online POMDP solving to address the uncertainty about a driver’s distraction in a shared control lane-keeping scenario while relying only on commonly available measures as observations.

\section{Related work and contributions}

% One important part I think is missing: the recent work in assist systems/agent for lane-keeping, AND the differences between yours.



POMDP is a framework to model sequential decision problems (one decision influences the next) in a partially observable environment \parencite{pomdp-definition}. An agent interacts with the environment. After performing an action, the agent receives observations about the environment and a reward signal. It uses the observations it receives over time to estimate the environment's state. Using its estimate of the true state of the environment, the agent can decide which actions to perform next to maximize the long-term reward.

POMDPs are well suited to model driving problems in which the internal psychological state of a human is important but unknown (such as her intend, goals, drowsiness, or attentiveness). Research mostly focuses on fully autonomous cars assessing the behavior of other traffic participants. By employing a POMDP to model the uncertainty about humans' internal states, improved driving policies were achieved in three, widely studied scenarios: pedestrian avoidance (\cite{despot-crowd}, and \cite{pomdp-pedestrian-avoid-2}), intersection crossing (\cite{pomdp-intersection}, \cite{att_intersec}, \cite{pomdp-intersection-2}, and \cite{pomdp-intersection-3}), and lane changes (\cite{pomdp-lane-changes}, \cite{att_intersec}, \cite{pomdp-lane-changes-2}, \cite{tactical-decision}, and \cite{pomdp_towards_human}). 

The aforementioned studies demonstrate that a POMDP is suitable for dealing with the uncertainty related to a person's internal psychological state and that this can be beneficial for making the right decisions when driving. However, the focus lies on estimating the state of external drivers rather than considering the state of the driver in a shared control scenario. \cite{hitl_pomdp} introduce a framework to model shared control driving scenarios with a human in the loop using POMDPs. The benefits of the framework are exemplified by assisting a potentially drowsy driver with lane-keeping: The assistance system is able to reason about the driver's drowsiness effectively and can handle noisy or erroneous observations. A driver model is used to simulate human behavior, and a simple car model is employed to simulate the car's dynamics in the experiments. However, the approach relies on the availability of observations that strongly correlate with the driver's drowsiness (observation whether the driver's eyes are closed or not).

Research suggests that driver distraction can lead to a change in driving performance. Distracted drivers show certain behaviors, such as lane position deviations, infrequent steering wheel movements, and reduced reaction times \parencite{driver-distraction-review}. \cite{dist-det-perf} show that monitoring driving performance measures (steering wheel angle, heading angle, and lateral deviation) alone can be sufficient to detect driver distraction accurately. Driven by those findings, we propose to estimate the driver's attentiveness in an online manner based on on driver's past driving behaviors.

In this thesis, an assisted lane-keeping scenario with a potentially distracted human in the loop is modeled as a POMDP. The requirement from \cite{hitl_pomdp} of observations that correlate very closely with the driver's psychological state is relaxed. The only observations available to the assistance system are the steering actions of the driver, the car's yaw angle, and its lateral deviation. Moreover, the number of considered discrete steering actions is substantially increased. The driving dynamics are simulated using a realistic driving simulator rather than a simple mathematical model.

\section{Problem overview and research questions}

The problem examined in this thesis is assisted lane keeping with shared control by a potentially distracted human driver and an agent acting as an assistance system. The agent assists the driver in keeping the car centered in its lane. The driver becomes distracted periodically. During distraction episodes, the driver may act suboptimally. The driver's behavior is simulated using a driver model. The driver’s attentiveness is unknown to the agent. It has to infer if the driver is distracted from the driver's past steering activity and sensory observations about the car's state. The dynamics of the car are simulated using the driving simulator \Gls{torcs}. A POMDP model is employed to account for the uncertainty about the driver's attentiveness.

\vspace{1em}
\noindent
\textbf{The following research questions arise:}
\begin{enumerate}
    \item How can a lane-keeping scenario with shared control between a potentially distracted human driver and an agent be modeled using a POMDP?
    % \item Which simplifications are necessary in order to address the problem?
    % \item How can a good policy for the agent be derived from the POMDP?
    % \item How does the agent perform with driver models of increasing complexity?
    % \item What measures are available to improve the agents performance?
    \item Can the agent estimate the driver's distraction using driver performance measures alone, allowing it to take appropriate actions when the driver becomes distracted?
    \item Is the solution approach viable for a real-world scenario, and if not, what limitations are there, and what are potential methods to solve them?
\end{enumerate}

\section{Outline}

The remaining content of the thesis is organized as follows:
\begin{description}
    \item[\cref{ch:theory}] introduces the basic theoretical \gls{pomdp} concepts which serve as a foundation for the thesis.

    \item[\cref{ch:problem}]
    presents the methodology, including the definition of the POMDP used to model the shared control lane-keeping task and the chosen solution approach.
    
    \item[\cref{ch:setup}]
    describes the experiments that we perform to assess the performance of the solution approach.
    
    \item[\cref{ch:results}]
    showcases the results from the experiments.

    \item[\cref{ch:discussion}]
    analyzes the results and elaborates the limitations of the solution approach.

    \item[\cref{ch:discussion}]
    highlights the conclusion, contributions, and challenges for a potential future real-world application.

\end{description}